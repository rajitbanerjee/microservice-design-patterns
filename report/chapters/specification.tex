\section{Initial Project Specification}

\subsection{Problem Statement}
Microservice architecture is a style of designing software systems to be highly maintainable, scalable, loosely coupled and independently deployable. Moreover, each service is built to be self-contained and implement a single business capability. Design patterns in software engineering refer to any general, repeatable or reusable solution to recurring problems faced during the software design process. This project aims to study the performance engineering practices associated with a number of \textit{microservice design patterns}, considering both qualitative and quantitative metrics to evaluate their benefits and shortcomings depending on the business requirement and use case. A non-exhaustive list of design patterns that could be explored is as follows:

\begin{itemize}
  \item API Gateway
  \item Asynchronous Messaging
  \item Chain of Responsibility
  \item Database or Shared Data
  \item Circuit Breaker
  \item Externalise Configuration
  \item Aggregator
  \item Branch
\end{itemize}

Employing some of the aforementioned design patterns, sufficiently complex simulations will be designed for the performance engineering experiments. The project will also look at some common issues in microservices, and how they compare with traditional monolithic architectures.

\subsection{Background}
Microservices have gained traction in recent years with the rise of Agile software development and a DevOps \cite{awsDevOps} approach. As software engineers migrate from monoliths to microservices, it is important to make appropriate choices for system design and avoid "anti-patterns". Although no one design pattern can be called the "best", the performance of systems can be optimised by following design patterns suited to the use case, with the right configuration of hardware resources.

\subsection{Related Work}
Due to their popularity, microservices have been written about extensively in books like \cite{richardson18}, \cite{kleppmann17}, \cite{newman14}. Articles such as \cite{md19}, \cite{md20}, \cite{sahiti20}, \cite{udantha19}, \cite{lewis14} discuss the intricacies of microservice architecture as well as the trade-offs between various common design patterns. In \cite{cully20}, the performance problems inherent to microservices are explored, with evaluations performed using a custom-built prototyping suite. Akbulut and Perros \cite{akbulut19} dive into the performance analysis aspect of microservices that is being proposed in this project, where they consider 3 different design patterns.

\subsection{Datasets}
Any data that is to be used or analysed in this project will be generated during the course of experiments. There are no dependencies on additional datasets.

\subsection{Resources Required}
A non-exhaustive list of resources is specified below, following a preliminary needs assessment.

\begin{itemize}
  \item Languages/Frameworks: Java, Spring Boot, Python
  \item Tools: Git, Docker, Apache JMeter
  \item Database: MongoDB
  \item Compute: Linux server maintained by the UCD School of Computer Science
\end{itemize}

