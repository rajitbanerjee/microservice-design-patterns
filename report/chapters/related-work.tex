\chapter{Related Work and Ideas}

A key task of this first report is to establish a baseline against which your later work will be judged. Your FYP project does not exist in a vacuum, and its central problem, or a variant thereof, will have been tackled by others before you. In this section, you should describe how previous approaches have tackled the problem, and clearly articulate the state of the art (or SOA) for your project.

For research-oriented projects, this task will be time-consuming but relatively straightforward. You should read past works on the subject, summarise the main points, pros and cons, and root out the previous works that they cite in turn. You may use Wikipedia as a secondary source only, which is to say that it can be a useful first port of call on many topics but not a source that should be liberally cited. Rather, use Wikipedia as a hub for gathering references to primary work in the field (original papers and reports), then read and summarise those. Do not quote a work that you have not read, unless you are quoting someone else’s view of that work. Never use another writer’s words as your own. Place any extracts from another’s work in double quotes, and attribute the quotation to its author with a citation. It is a very low act to plagiarise another’s work and take credit for their words, so tread carefully. Even unintentional plagiarism is still plagiarism.

For more application-oriented projects, you are still expected to survey other solutions, either for the given problem or for similar problems, and also consider applications that share functionality or design principles with your own. In short, this section is the core of your report regardless of what kind of project you do.
