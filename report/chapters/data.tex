\chapter{Data Considerations}

In this section you should characterise the nature and scale of the data you are working with. Outline the shape of the data, where you expect to obtain it, and the size of the data. Is it static or dynamic, local or remote, stored or streaming? Is it raw or structured? Is it unfiltered user data, or is it curated by a specialist? What is your rationale for using this data and not other data? If your project looks at callout times for Spanish ambulances, usage rates of French parking lots, alcohol consumption in Germany, and so on, then explain why you are not using Irish data for the project. Indicate the data-cleaning processes that you anticipate will be necessary. What licensing restrictions, if any, apply to your data? Will you be making this data public after your project is completed? Are there any privacy or ethical issues with how the data is to be collected or used? If so, discuss here.

Some or many of these questions may be moot in the case of specific projects, but you should provide compelling answers to any that seem relevant. Since this provides the foundation for your project, your reviewers will be looking closely.
