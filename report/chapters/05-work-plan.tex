\chapter{Project Work Plan}

In this section you will present a work plan for the remainder of your project. Show that you have considered the issues carefully, and that you can be trusted to lead a research or development effort. Be as specific as you can about the time you expect to allocate to each work component, and the dependencies they have to each other. A Gantt chart is helpful in this respect, but do show some sense in how you present your plan. A naïve understanding makes for a simplistic plan.

A key part of a successful project is evaluation. It is not enough to just state that your project is a success, or that your friends seem to like it. You must have a plan for evaluating the end result. How you evaluate will depend on the nature of your project, and you should have a serious conversation with your supervisor about evaluation before you get to this stage. Will your work yield quantitative results that can be compared to past work or to established benchmarks? Does your work consider different configurations of a system or a solution that you can compare to each other, allowing you to empirically find the best one? Do you have a sample user pool for your planned application, and are they willing to give you structured qualitative and quantitative feedback (e.g. via a questionnaire)? However you plan to evaluate your project, please sketch your intentions here.
