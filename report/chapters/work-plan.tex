\section{Project Work Plan}

\begin{figure}[H]
  \centering
  \includegraphics[width=0.9\linewidth]{./assets/images/work-plan-gantt}
  \caption{Gantt chart for project timeline.}
  \label{fig:work-plan-gantt}
\end{figure}

\begin{itemize}
  \item \textbf{Design Patterns Planning (1 week)}: The initial planning here is of particular significance and will decide the direction and pace of the project's implementation phase. Various microservice design patterns will be considered to select a few important patterns which can be easily demonstrated using simulations.

  \item \textbf{Testing Framework Design (1 week)}: Designing a simple testing framework (e.g. load testing plans) for the first case study will facilitate the adoption of similar strategies for subsequent case studies.

  \item \textbf{Case Studies (Parts I, II, III, IV) (8 weeks)}: These case studies will form the bulk of the project, and will be split into four parts for convenience, each comprising a set of related design patterns (each group of case studies will possibly address 2-3 patterns). The majority of effort required here will be concerned with the back-end development of dummy microservice-based systems using containers (Docker). Evaluation, both qualitative and quantitative, is well integrated with the implementation phase of the project, since performance analysis/testing will be carried out in tandem with the case study experiments.

  \item \textbf{Final Report (3 weeks)}: Although the core parts of the report should be written as progress is made with tasks, a dedicated period is set aside for refinement and completion.

  \item \textbf{Contingency (2 weeks)}: Time set aside to be used only in the event of unforeseen issues or challenges.
\end{itemize}

